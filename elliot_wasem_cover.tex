\documentclass[12pt,letterpaper,roman]{moderncv}
\usepackage[scale=0.8,margin=1.5in]{geometry}
\newgeometry{top=1.0in, right=1.0in, bottom=1.0in, left=1.0in}

\moderncvstyle{classic}
\moderncvcolor{blue}

\newcommand*{\mysection}[1]{%
  \vspace{2.5ex}
  \phantomsection{}
  \addcontentsline{toc}{part}{#1}%
  \parbox[m]{\textwidth}{\sectionstyle{#1}}\\[1ex]}

\name{Elliot}{Wasem}
\title{Software Engineer}
\address{Lyndhurst, NJ, USA}
\phone[mobile]{1-609-498-2128}
\email{elliotbielwasem@gmail.com}
\homepage{elliotwasem.xyz}
\social[linkedin]{elliot-wasem}
\social[github]{elliot-wasem}

\begin{document}

\maketitle{}

\sethintscolumntowidth{\ }
Hello,\bigskip

I am a computer science major earning my Bachelor's of Science from Stevens Institute of Technology in May of 2020, and am seeking employment beginning as soon after my May graduation date as possible.\bigskip

As a software developer, I take a strong stance in favor of stability, scalability, and efficiency over quick implementation, and it shows in my code. I am extremely strong in low-level development, and my in-depth knowledge of UNIX or UNIX-like (BSD, Linux, Solaris, etc.) systems and how their kernels work allows me to take advantage of stability and performance optimizations that otherwise may be overlooked. I have experience writing enterprise-level software, and feel comfortable with creating scalable solutions that is just as efficient running in a single node as communicating across distributed systems.\bigskip

That is not to say that I am naive enough to think I know everything. I regularly come across new topics for which I have no prior expertise off which to work. This means that I am constantly learning and growing, and feel very comfortable learning on the job, and considering many different plans of attack before moving forward with a solution.\bigskip

I look forward to hearing more from you all, and answering any further questions you may have.\bigskip

Best,

Elliot Wasem
\end{document}
